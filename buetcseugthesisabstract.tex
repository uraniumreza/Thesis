% Do not change these lines
\renewcommand{\abstractname}{\textbf{{\Large ABSTRACT}}}
\addcontentsline{toc}{chapter}{\textbf{\normalsize{\emph{ABSTRACT}}}}

% You need to change the text for abstract
\begin{abstract}\thispagestyle{plain}
Determining the optimal number of clusters in a data set is a fundamental issue in partitioning clustering, such as $K$-means clustering, which requires the user to specify the number of clusters $K$ before the algorithm is applied. Unfortunately, there is no definitive approach to find out optimal number of clusters, $K$, for any given dataset. The optimal number of clusters is somehow subjective and depends on the method used for measuring similarities and the parameters used for partitioning. In this thesis work, we will describe different methods for determining the optimal number of clusters for $K$-mean clustering. These methods include direct methods and statistical testing methods. Direct methods consist of optimizing a criterion, such as the within cluster sums of squares or the average silhouette. The corresponding methods are named elbow and silhouette methods, respectively. Statistical testing methods consist of comparing evidence against null hypothesis. And we will use the gap statistic method. Then we have applied these three existing methods for selecting the number of clusters, $K$, for $K$-means clustering on various kind of datasets with different characteristics and compared the results with the original number of clusters. The dissertation concludes with a performance analysis of the results.
\end{abstract}

\endinput
