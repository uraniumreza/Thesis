% Do not change these lines
\renewcommand{\abstractname}{\textbf{{\Large ABSTRACT}}}
\addcontentsline{toc}{chapter}{\textbf{\normalsize{\emph{ABSTRACT}}}}

% You need to change the text for abstract
\begin{abstract}\thispagestyle{plain}
The $K$-means algorithm  is  a  popular data clustering  algorithm.  However,  one  of  its major drawbacks is the requirement for the number of clusters, $K$, to be specified before the algorithm is applied. And the performance of $K$-means clustering mostly depend on the determining number of clusters of a given dataset. In this thesis work we have studied direct and statistical methods for finding number of clusters. The chosen direct methods are Elbow Method and Average Silhouette Method; statistical method is Gap Statistic Method. Here we have applied these three most important existing methods for selecting the number of clusters $K$ for $K$-means algorithm on various kind of datasets with different characteristics. Then we have compared the outputs of different methods with the original number of clusters. And finally we have tried to find out the performance of these methods to find the optimal number of clusters on different datasets. The dissertation concludes with an analysis of the results of using the methods to determine the number of clusters for the $K$-means algorithm.
\end{abstract}

\endinput
