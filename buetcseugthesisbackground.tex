\chapter{Background}\label{background}

\section{Introduction}
\subsection{Machine Learning}
Machine learning is the subfield of computer science that gives computers the ability to learn
without being explicitly programmed. Evolved from the study of pattern recognition and computational
learning theory in artificial intelligence, machine learning explores the study and
construction of algorithms that can learn from and make predictions on data such algorithms
overcome following strictly static program instructions by making data driven predictions or
decisions, through building a model from sample inputs. Machine learning is employed in a
range of computing tasks where designing and programming explicit algorithms is infeasible;
example applications include spam filtering, detection of network intruders or malicious insiders
working towards a data breach, optical character recognition (OCR), search engines and
computer vision.\\

Tom M. Mitchell~\cite{Michalski2013ml} provided a widely quoted, more formal definition:

``A computer program is said to learn from experience E with respect to some class
of tasks T and performance measure P if its performance at tasks in T, as measured
by P, improves with experience E."

Machine learning tasks are typically classified into three broad categories, depending on the
nature of the learning ``signal" or ``feedback" available to a learning system. These are:\\

\textbf{Supervised Learning}\\
Supervised learning is where you have input variables ($x$) and an output variable ($Y$) and
you use an algorithm to learn the mapping function from the input to the output. Exact functon
may be as $Y$ = $f(X)$. The goal is to approximate the mapping function so well that when you
have new input data ($x$) that you can predict the output variables ($Y$) for that data.

It is called supervised learning because the process of an algorithm learning from the training
dataset can be thought of as a teacher supervising the learning process. We know the correct
answers, the algorithm iteratively makes predictions on the training data and is corrected by the
teacher. Learning stops when the algorithm achieves an acceptable level of performance. Supervised
learning problems can be further grouped into regression and classification problems.

\begin{itemize}
\item \textbf{Classification} : A classification problem is when the output variable is a category, such as
``red" or ``blue" or ``disease" and ``no disease".
\item \textbf{Regression} : A regression problem is when the output variable is a real value, such as
``dollars" or ``weight".\\
\end{itemize}

\textbf{Unsupervised Learning}\\
No labels are given to the learning algorithm, leaving it on its own to find structure in its
input. Unsupervised learning can be a goal in itself (discovering hidden patterns in data) or a
means towards an end (feature learning).Unsupervised learning is where you only have input
data $(X)$ and no corresponding output variables.

The goal for unsupervised learning is to model the underlying structure or distribution in the
data in order to learn more about the data.These are called unsupervised learning because unlike
supervised learning above there is no correct answers and there is no teacher. Algorithms are left
to their own devises to discover and present the interesting structure in the data.Unsupervised
learning problems can be further grouped into clustering and association problems.

\begin{itemize}
\item \textbf{Clustering} : A clustering problem is where you want to discover the inherent groupings
in the data, such as grouping customers by purchasing behavior.
\item \textbf{Association} : An association rule learning problem is where you want to discover rules
that describe large portions of your data, such as people that buy X also tend to buy Y.\\
\end{itemize}

\textbf{Semi-Supervised Machine Learning}\\
Problems where there is a large amount of input data (X) and only some of the data is labeled
(Y) are called semi-supervised learning problems.These problems sit in between both supervised
and unsupervised learning. A good example is a photo archive where only some of the
images are labeled, (e.g. dog, cat, person) and the majority are unlabeled.

Many real world machine learning problems fall into this area. This is because it can be
expensive or time-consuming to label data as it may require access to domain experts. Whereas
unlabeled data is cheap and easy to collect and store.\\

\textbf{Reinforcement Learning}\\
A computer program interacts with a dynamic environment in which it must perform a certain
goal (such as driving a vehicle or playing a game against an opponent). The program is provided
feedback in terms of rewards and punishments as it navigates its problem space.

\section{A Deep Dive into Cluster Analysis}
Cluster analysis groups data objects based only on information found in the
data that describes the objects and their relationships.  The goal is that the
objects within a group be similar (or related) to one another and different from
(or unrelated to) the objects in other groups.  The greater the similarity (or
homogeneity) within a group and the greater the difference between groups,
the better or more distinct the clustering.

Cluster analysis is related to other techniques that are used to divide data
objects into groups.  For instance, clustering can be regarded as a form of
classification in that it creates a labeling of objects with class (cluster) labels.
However, it derives these labels only from the data. In contrast, classification
in the sense of our previous section is \textbf{supervised classification}; i.e.,
new, unlabeled objects are assigned a class label using a model developed from objects with
known class labels. For this reason, cluster analysis is sometimes referred to as
\textbf{unsupervised classification}. When the term classification is used
without any qualification within data mining, it typically refers to supervised
classification.

\section{A}
\section{B}
\section{C}
\section{D}

\endinput
