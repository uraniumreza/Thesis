\chapter{Conclusion and Future Work}\label{ch:conclusion}

\section{Introduction}
In this beautiful and mysterious world there is nothing as to be final and accurate or flawless.
There must be some pitfalls or crevices to everything. This is the ultimate rule and this is
the inherent daintiness. There was a time when everyone believes that mass and time are real
constant and cannot have different value at different circumstances in the era of Newton. Then
there comes Einstein with his great mind and changed the way of thinking of whole world. He
had shown how these very obvious things can be realative in different context of speed. Once
upon a time the shape of the mother earth is considered as a flat land and center of our solar
system. But now what we have learnt by the great minds of our history is the rounding shape of
the mother earth and nothing but a tiny planet of the galaxy. We believe all these things without
seeing the fact with our own eyes because of mathmetical proof. Once Albert Einstein said that
“Pure mathematics is, in its way, the poetry of logical ideas”.

It is also true about our improvement of the determining the number of clusters. There have so
many different methods exist. Different minds have different directions of view. We have
tried to highlight the three major methods of finding $K$ for $K$-means algorithm.

\section{Conclusion}
Existing methods of selecting the number of clusters for
$K$-means clustering have a number of drawbacks.
Also,  current  methods  for  assessing  the  clustering
results  do  not  provide   much  information   on  the
performance of the clustering algorithm.
Three  methods  to  select  the  number  of  clusters
for  the $K$-means  algorithm  have  been  reviewed  in
this work. But we did not propose any new method or improvement
of existing methods.

\section{Future Work}
In future we like to add some more datasets in our simulation process
and compare them with more different methods of selecting number of
clusters for $K$-means algorithm. And thus we hope that we can propose a new way of finding
number of clusters efficiently and optimally for any given dataset.

\endinput
